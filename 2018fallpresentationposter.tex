\documentclass[12pt]{article}
\usepackage{fullpage}
\usepackage{wallpaper}
\usepackage{color}
\pagestyle{empty}

\begin{document}
\ThisCenterWallPaper{1}{gh.pdf}
\begin{center}
\sffamily
\Huge\textbf{GRADUATE SEMINAR}

\bigskip
\LARGE\textbf{David Luke Thiessen}

\bigskip
\textbf{Models for Non-Monotone Missing at Random Data}

\bigskip
\large
\textit{PhD Student supervised by Yang Zhao}
%or
%\textit{MSc Student supervised by}

\Large
\bigskip
\textbf{October 15, 2018
\\3:30 p.m.
\\Math Lounge
}
\end{center}

\bigskip

\sffamily\large\noindent\textbf{Abstract:} 
One difficulty that arises in applied statistics is the presence of missing data, when the sample design calls for data to be collected but the researchers are unable to. When the mechanism that creates the missingness is correlated with the variable under study, the sample is no longer representative of the whole population. The difference between the sample and the population means that results from the statistical analysis may be biased and not extend to the entire population. In this talk we provide a basic framework to carry out statistical analysis when data are missing and demonstrate how certain techniques can be used to reduce or eliminate the bias caused by missingness.


 \end{document}